1. Selezionare tutti gli studenti iscritti al Corso di Laurea in Economia
SELECT `degrees`.`name`,`students`.`name` FROM `degrees` iNNER JOIN `students` ON `students`.`degree_id` = `degrees`.`id` WHERE `degrees`.`name`='Corso di Laurea in Economia';


2. Selezionare tutti i Corsi di Laurea Magistrale del Dipartimento di Neuroscienze
SELECT `departments`.`name`,`degrees`.`level` FROM `departments` INNER JOIN `degrees` ON `degrees`.`department_id` = `departments`.`id` WHERE `departments`.`name` = 'Dipartimento di Neuroscienze' AND `degrees`.`level`='Magistrale';

3. Selezionare tutti i corsi in cui insegna Fulvio Amato (id=44)
SELECT DISTINCT `teachers`.`name`,`teachers`.`surname`,`courses`.`name` FROM `teachers` INNER JOIN `course_teacher` ON `teachers`.`id` = `course_teacher`. `teacher_id` JOIN `courses` ON `teachers`.`id` = `courses`.`degree_id` WHERE `teachers`.`id`=44;

4. Selezionare tutti gli studenti con i dati relativi al corso di laurea a cui sono iscritti e il
relativo dipartimento, in ordine alfabetico per cognome e nome
SELECT DISTINCT `departments`.`name` AS departments_name,`degrees`.* ,`students`.`surname` AS surname, `students`.`name` AS name FROM `departments` INNER JOIN `degrees` ON `departments`.`id` = `degrees`.`department_id` INNER JOIN `students` ON `degrees`.`id` = `students`.`degree_id` ORDER BY `students`.`surname` ASC, `students`.`name` ASC;

5. Selezionare tutti i corsi di laurea con i relativi corsi e insegnanti



6. Selezionare tutti i docenti che insegnano nel Dipartimento di Matematica (54)


7. BONUS: Selezionare per ogni studente quanti tentativi d’esame ha sostenuto per
superare ciascuno dei suoi esami